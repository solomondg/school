\documentclass[12pt]{article}
\usepackage{amsmath}
\usepackage{fancyhdr}
\usepackage{geometry}
\usepackage{parskip}
\usepackage{pdfpages}
\usepackage{graphicx}
\graphicspath{{./}}
\geometry{letterpaper, portrait, margin=1in}
\setlength{\parindent}{0pt}

\title{Final FRQs}
\date{2016/05/24}
\author{Solomon Greenberg}

\fancyhf{}
\pagestyle{fancy}

\lhead{Final FRQs}
\rhead{Solomon Greenberg}



\begin{document}
    \pagenumbering{gobble}
    \newpage
    \pagenumbering{arabic}
    \paragraph*{FRQ 1:} Bacterial resistance
        \subparagraph*{(a)}
            \begin{itemize}
                \item The most effective antibiotic would be Imipenem. Despite Tazobactam resistance having proliferated less since 1994, Imipenem is still less resisted.
                \item The least effective antibiotic would be Ampicillin. Despite more bacteria becoming Ciprofloxacin resistant since 1994, Ampicillin is more resisted.
            \end{itemize}
        \subparagraph*{(b)\\} Over just 6 years, bacterial infections became 10\% more resistant to Ciprofloxacin and 6\% more resistant to Gentamicin.
        \subparagraph*{(c)}
        \begin{enumerate}
                \item Mutations can happen in individual bacteria due to radiation or simple DNA replication or transcription errors.
                \item Bacteria with the necessary genetic code for antibiotic resistance can \textit{conjugate} the code, packaging it into the form of a plasmid and transferring it to another bacteria.
                \item As well, bacteria can simply uptake free-floating DNA or have it injected into it by a virus; it would be evolutionary favorable for a virus to strengthen its host.
        \end{enumerate}
        \subparagraph*{(d)}
        \begin{enumerate}
            \item Natural selection selects for those who can pass on their genes. As such, bacteria that survives (and does not get killed by antibiotics) are much more successful in replicating.
            \item Genetics are heridetary, so, i.e.\, antibiotic resistance (being a property of genetics) are passed down generations, allowing natural selection to happen and for the offspring of successful bacteria to be successful themselves
        \end{enumerate}
            Evolution is happening, and we can see this due to the proliferation of antibiotic-resistant bacteria. Using Hardy-Weinberg, if antibiotic resistance is autosomal recessive, then it's $q^2$. If 14\% of bacteria was resistant to Gentamicin in 1994, $q^2 = 0.14, q = \sqrt{0.14} = 0.37, p = 1 - 0.37 = 0.63$, so $p = 0.37$ and $q = 0.63$. Now, in 2000, $q^2 = 0.22, q = \sqrt{0.22} = 0.47$, so $p = 0.53$. So, $p$ changed from $0.37$ to $0.47$, and $q$ changed from $0.63$ to $0.53$. As these two values changed, we know that evolution has happened.

        \subparagraph*{(e)\\}
            An ethical question would be whether or not it is ethically right to give antibiotics to beef cattle while knowing that it contributes to antibiotic resistance and therefore can put lives at risk, especially moving towards the future.

    \paragraph{FRQ 2:} Cell Cycle Phases
        \subparagraph*{(a)}
            \begin{enumerate}
                \item G1 Stage: This stage is first, and it's where the cell starts to break down sugars into energy (glycolysis). The DNA is still in chromosone form, i.e.\ only 1 DNA molecule
                \item S Stage: This stage is second, and it's where the cell replicates DNA, turning each chromosone to two chromatids
                \item G2 Stage: This stage is third, and it's where the cell synthesizes more organelles. Final stage before meitosis or meiosis (depending on the cell).
            \end{enumerate}
        \subparagraph*{(b)}
         EGF most likely stimulates local skin cells to replicate. When a skin (dermal/dermis) cell's EGFR receptors (the protiens in their membrane) are activated by EGF, the most likely result is a phosphorylation cascade, resulting in gene expression or cell replication.
        \subparagraph*{(c)}
         If the EGFR receptor amount is increased, the cells become much more sensative to EGF, causing unwanted growth.
        \subparagraph*{(d)}
         The antibody would bind with the EGFR receptor but not activate it, bringing EGF sensitivity down to a normal or subnormal level. This would be effective for treating cancer brought about by an oversensitivity to EGF.


    \paragraph{FRQ 3:} Corn Seedlings
        \subparagraph*{(a)} One inorganic source is $\mathrm{CO_{2}}$. It becomes G3P, i.e.\ glucose, in the Calvin cycle in photosynthesis.
        \subparagraph*{(b)} The chemical reaction is $\mathrm{6CO_{2} + 6H_{2}O} \xrightarrow{sunlight} \mathrm{C_{6}H_{12}O_{6} + 6O_{2}}$. As there was no sunlight with the dark plants, sugar couldn't be produced, so internal stores of sugar were required as the light cycle in photosynthesis couldn't function.

    \paragraph{FRQ 4:} Stimulus on motor neuron
        \subparagraph*{(a)} The $\mathrm{K^+}$ graph has a much lower peak permeability than the $\mathrm{Na^+}$ graph, and takes slightly longer to reach said peak, but the permeability lasts much longer.
        \subparagraph*{(b)} The voltage-gated ion channels regulate permeability
        \subparagraph*{(c)} The $\mathrm{Na^+}$ graph correlates closely with the membrane potential graph.

    \paragraph{FRQ 5:} Blood lactate levels
        \subparagraph*{(a)} As Weddell Seals dive over the roughly 20 minute mark, they run out of stored oxygen in their lungs, so lactate in their tissues travels into the bloodstreamand is not expelled, raising blood lactate levels.
        \subparagraph*{(b)} Chances are, it would be slowly and steadily decreasing until the 20 minute mark, then would decrease rapidly.
        \subparagraph*{(c)} Evolutionarily, Emperor Penguins may hunt prey that is easier to find or closer to the surface of the water, while Baikal Seals may hunt harder to find prey or prey that is deeper under water, putting evolutionary pressure on the Baikal Seals for longer dive times, but not the Penguins to such a degree.

    \paragraph{FRQ 6:} Polypeptides
        \subparagraph*{(a)} Secretory protiens are secreted from the rough endoplasmic reticulum, where they are packaged into vescicles by the Golgi Apparatus, then secreted.
        \subparagraph*{(b)} RNA is transcribed in the nucleus then translated in the rough ER, so I'd predict that the radioactive labeling would originate in the nucleus and travel to the rough ER.

    \paragraph{FRQ 7:} Distribution of plants\\
        Plants could be moving to fulfill a previously-unfulfilled ecological niche, or to an area with more abundant resources. THe evidence would be the amount of water and sunlight on lower vs higher areas. There's not much reason for plants to shift other than more resource availability.

    \paragraph{FRQ 8:} Enzyme-catalyzed reaction
        \subparagraph*{(a)} The primary structure (i.e.\ the sequence of amino acids) directly influences the shape by changing how the protein can fold and bond with itself.
        \subparagraph*{(b)} At a high substrate concentration, a noncompetitive inhibitor would reduce the rate of reaction. Noncompetitive inhibitors bind to enzymes, reducing their effectiveness or halting them completely, thus lowering the reaction rate.


\thispagestyle{fancy}

\end{document}

