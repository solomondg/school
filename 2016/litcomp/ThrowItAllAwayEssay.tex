% This file was converted to LaTeX by Writer2LaTeX ver. 1.4
% see http://writer2latex.sourceforge.net for more info
\documentclass[letterpaper]{article}
\usepackage[latin1]{inputenc}
\usepackage[T1]{fontenc}
\usepackage[english]{babel}
\usepackage{amsmath}
\usepackage{amssymb,amsfonts,textcomp}
\usepackage{color}
\usepackage{array}
\usepackage{hhline}
\usepackage{hyperref}
\hypersetup{pdftex, colorlinks=true, linkcolor=blue, citecolor=blue, filecolor=blue, urlcolor=blue, pdftitle=, pdfauthor=, pdfsubject=, pdfkeywords=}
% Page layout (geometry)
\setlength\voffset{-1in}
\setlength\hoffset{-1in}
\setlength\topmargin{0.7874in}
\setlength\oddsidemargin{0.7874in}
\setlength\textheight{9.4251995in}
\setlength\textwidth{6.9251995in}
\setlength\footskip{0.0cm}
\setlength\headheight{0cm}
\setlength\headsep{0cm}
% Footnote rule
\setlength{\skip\footins}{0.0469in}
\renewcommand\footnoterule{\vspace*{-0.0071in}\setlength\leftskip{0pt}\setlength\rightskip{0pt plus 1fil}\noindent\textcolor{black}{\rule{0.25\columnwidth}{0.0071in}}\vspace*{0.0398in}}
% Pages styles
\makeatletter
\newcommand\ps@Standard{
  \renewcommand\@oddhead{}
  \renewcommand\@evenhead{}
  \renewcommand\@oddfoot{}
  \renewcommand\@evenfoot{}
  \renewcommand\thepage{\arabic{page}}
}
\makeatother
\pagestyle{Standard}
\title{}
\author{}
\date{2016-04-13}
\begin{document}

\bigskip

\ \ The song ``Throw It All Away, `` by Toad the Wet Sprocket, features the narrator advocating to the subject to value
friends and to try to follow what feels like their true purpose is in life. However, the subject of the song is
ambiguous, and could be easily either society as a whole or a single person the singer is addressing directly.

\ \ Overall, there are phrases with ambiguous subjects, and more specific ones that seem to be speaking to only one
person, which I'd argue is what the song is aiming to do. Close to the beginning of the song, there's the lines ``Take
your cautionary tales/And take your incremental gain/And all the sycophantic games/And throw `em all away.'' I'd say
that while this can be applied to society as a whole, the specificness of ``your'' combined with that of the three
subjects indicates a more personal touch. As well, towards the end of the song, the line ``...fill the hole insdie your
heart'' further supports the personal motif, with it -- arguably -- requiring more personal knowledge of a person 
\end{document}
