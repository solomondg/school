
\documentclass[12pt]{article}

\usepackage{graphicx}
\usepackage{natbib}
\usepackage{siunitx}
\usepackage{geometry}
\usepackage{fancyhdr}

\graphicspath{{./}}
\geometry{letterpaper, portrait, margin=1in}

\setlength\parindent{18ex}


\title{Investigating the Relation \\ Between Variables in the \\ Ideal Gas Law Equation}

\date{2016/10/16}
\author{Solomon Greenberg}

\fancyhf{}
\pagestyle{fancy}

\begin{document}


\maketitle

\begin{center}
\begin{tabular}{l r}
\end{tabular}
\end{center}

\section{Abstract}
We investigated relationships between the pressure, volume, moles, and temperature of ideal gasses in a computer-based simulation environment. With non-mentioned properties remaining constant in each case, there was an inverse relationship between pressure and volume, a direct relationship between pressure and mols of gas, direct between pressure and temperature, and direct between volume and temperature, confirming the hypothesis. $P < 0.05$ and $R > 0.98$ for all cases. (68)


\section{Introduction}
This lab aimed to investigate the relationship between variables in the ideal gas law equation. In specific, the equation is $PV=nRT$. In this equation, the variables for pressure—($P$, Pa), volume ($V$, cm$^3$), moles of gas ($n$, mole), and temperature ($T$, K)—are all related to each other. $R$ is a constant, so it can be ignored. It is expected (hypothesized) that, with all non-mentioned variables held constant in each case, there will be an inverse relationship between $P$ and $V$ and a direct relationship between $P$ and $n$, $P$ and $T$, $V$ and $T$, and $V$ and $n$. Since pressure is simply force times area (component of volume), it is logically reasoned that as area (volume) increases, pressure goes down, prompting the inverse relationship. Similarly, one can reason that as one increases the number of moles of gas, there will be more collisions per unit time against the walls of the container, prompting the direct relationship between $n$ and $P$. The relationship between $P$ and $T$ can be explained through the fact that as temperature (average particle velocity) rises, the kinetic energy exerted on the container (pressure) must rise as well. The relationship between $V$ and $T$ can be most closely likened to that between $P$ and $T$ and $P$ and $V$. If we wish to hold pressure constant while increasing temperatures, we must raise volume (decrease volume when decreasing temperatures), as pressure is force per unit area, and is thus inverse to volume.


\section{Experimental Design}
The PhET Gas Properties simulation was the main tool for data collection used in this lab. For each relation tested, all non-related variables were marked to be held constant. For each relation, 3-5 data points were then taken, with varying values of the independent variable.

\section{Results}
Our data confirmed our hypothesis, finding an inverse relationship between pressure and volume, and direct between pressure and both moles and temperature, and volume and temperature.



\end{document}
