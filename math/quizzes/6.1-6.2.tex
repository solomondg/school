\documentclass[12pt]{article}
\usepackage{amsmath}
\usepackage{fancyhdr}
\usepackage{geometry}
\usepackage{parskip}
\usepackage{pdfpages}
\usepackage{graphicx}
\usepackage{mathtools}

\graphicspath{{./}}
\geometry{letterpaper, portrait, margin=1in}
\setlength{\parindent}{0pt}

\fancyhf{}
\pagestyle{fancy}

\lhead{Math 252 Take-home Quiz}
\rhead{Solomon Greenberg}

\newcommand{\me}{\mathrm{e}}
\newcommand{\dx}{\mathrm{d}x}
\newcommand{\du}{\mathrm{d}u}
\newcommand{\dtheta}{\mathrm{d}\theta}
\newcommand{\md}{\mathrm{d}}
\DeclarePairedDelimiter\abs{\lvert}{\rvert}%
\DeclarePairedDelimiter\norm{\lVert}{\rVert}%

% Swap the definition of \abs* and \norm*, so that \abs
% and \norm resizes the size of the brackets, and the 
% starred version does not.
\makeatletter
\let\oldabs\abs
\def\abs{\@ifstar{\oldabs}{\oldabs*}}

\let\oldnorm\norm
\def\norm{\@ifstar{\oldnorm}{\oldnorm*}}
\makeatother


\begin{document}
    \pagenumbering{gobble}
    \newpage
    \pagenumbering{arabic}

    % \paragraph*{6.1:} 2, 3, 6, 9, 15, 17, 21, 22, 23, 26, 27, 29, 30
    % \paragraph*{6.2:} 1, 2, 3, 5, 9, 14, 16, 20, 21, 25, 26, 28, 31, 33

    \paragraph*{1:\\}
    $\int_{-1}^{2} \! \frac{x}{\sqrt{x^2 + 1}} \, \dx$\\
    Need to use $u$-substitution\\
    $u = x^2 + 1$\\
    $f(x) = \frac{1}{\sqrt{g(x)}}$\\
    $\du = 2x\dx$\\
    $x\dx = \frac{\du}{2}$\\
    $\int \! \frac{1}{2\sqrt{u}}\,\du$\\
    $= \frac{1}{2} \int \! \frac{1}{2\sqrt{u}}\,\du$\\
    $= \frac{1}{2} \int \! u^{-1/2} \, \du $\\
    $= \frac{1}{2} \cdot 2 \sqrt{u}$\\
    $= \sqrt{u}$\\
    $\sqrt{u} = \sqrt{x^2 + 1}$\\
    $\int \! \frac{x}{\sqrt{x^2 + 1}}\dx = \sqrt{x^2 + 1} (+ C)$\\
    Between -1 and 2:
    $= \sqrt{5} - \sqrt{2}$\\

    \paragraph{2:\\}
    $\int \! x\me^{-4x} \, \dx$\\
    $u = \me^{-4x}, v = \frac{x^2}{2}$\\
    $u' = -4\me^{-4x}, v' = x$\\
    $uv - \int \! vu'$\\
    $\frac{x^2\me^{-4x}}{2} - \int \frac{-4x^2\me^{-4x}}{2}$\\
    $=\frac{x^2\me^{-4x}}{2} - \frac{(8x^2 + 4x + 1)\cdot \me^{-4x}}{16}$\\
    $=\frac{-(4x + 1)\cdot\me^{-4x}}{16} + C$\\

    \paragraph{3:\\}
    $\int \! \frac{14}{(2x-1)(x+3)}\,\dx$\\
    $\frac{14}{(2x-1)(x+3)} = \frac{A}{2x-1} + \frac{B}{x+3}$\\
    $14 = A(x+3) + B(2x-1)$\\\\
    To remove A:\\
        $x = -3$\\
    $14 = A(0) = B(-7)$\\
    $-14 = 7B$\\
    $B = -2$\\\\
    To remove B:\\
        $x = \frac{1}{2}$\\
    $14 = A(\frac{7}{2}) + B(0)$
    $14 = \frac{7A}{2}$\\
    $7A = 28$\\
    $A = 4$\\
    $A = 4, B = -2$\\

    $\int \! \frac{14}{(2x-1)(x+3)}\,\dx = \int\!\frac{4}{2x-1}\,\dx + \int\!\frac{-2}{x+3}\,\dx$\\
    $= 2\ln{\abs{2x-1}} + -2\ln{\abs{x+3}} + C$\\

    \paragraph*{4:\\}
    $\int_{1}^{3}\!\ln{x}\dx$\\
    Int by parts needed.\\
    $\int\!uv' = uv - \int \! vu'$\\
    $u = \ln{x}, v = x$\\
    $u' = \frac{1}{x}, v' = 1$\\
    $x\ln{x} - \int\! 1$\\
    $= x\ln{x} - x$\\
    Between 1 and 3\\
    $(3\ln{3} - 3) - (\ln{1} - 1)$\\
    $= 3\ln{3} - 2$\\

    \paragraph*{5:\\}
    $\int\!\frac{1}{\sqrt{81-x^2}}\,\dx$\\
    $x = 9\sin{u}$\\
    $\dx = 9\cos{u}\du$\\
    $= \frac{1}{\sqrt{-81\sin^2{} + 81}}$\\
    $=\frac{1}{9\sqrt{-\sin^2{u}+1}}$\\
    $=\frac{1}{9\sqrt{\cos^2{u}}}$\\
    $=\frac{1}{9\cos{u}}$
    $u = \arcsin{\frac{x}{9}}$\\
    $\int\!x \ln{x^2} \,\dx = u$\\
    $= \arcsin{\frac{x}{9}} + C$\\

    \paragraph*{6:\\}
    $\int\!x \ln{x^2} \,\dx$\\
    U-sub\\
    $u = x^2$\\
    $\frac{\du}{2} = x\dx$\\
    $=\int\!\frac{\ln{u}}{2}\,\du$\\
    Int by parts:\\
    $\int\!ab' = ab - \int\!ba'$\\
    $a = \ln{u}, b = \frac{u}{2}$\\
    $a' = \frac{1}{u}, b' = \frac{1}{2}$\\
    $= \frac{u\ln{u}}{2} - \int\!\frac{1}{u}\frac{u}{2}$\\
    $= \frac{u\ln{u}-u}{2}$\\
    $= \frac{x^2(\ln{x^2}-1)}{2} + C$


    \paragraph*{7:\\}
    $\int\!\arctan{x}\,\dx$\\
    $u = \arctan{x}, v = x$\\
    $u' = \frac{1}{x^2+1}, v' = 1$\\
    $x\arctan{x} - \int\!\frac{x}{x^2+1}$\\
    U-sub:\\
    $u = x^2 + 1$\\
    $\du = 2x\dx$\\
    $x\dx = \frac{\du}{2}$\\
    $x\arctan{x} - \frac{1}{2} \int\!\frac{1}{u}\,\du$\\
    $x\arctan{x} - \frac{1}{2}\ln{\abs{u}} = x\arctan{x} - \frac{1}{2}\ln{\abs{x^2 + 1}}$\\
    $= x\arctan{x} - \frac{1}{2}\ln{\abs{x^2 + 1}} + C$\\


    \paragraph*{8:\\}
    $\int\!\frac{1}{(4x^2+9)^2}$\\
    Let $x = \frac{3}{2}\tan{\theta}$\\
    $\dtheta =$

    \paragraph*{9:\\}
    $\int\!\sin^2{\theta}\cos{\theta}\,\dtheta$
    $u = \sin{x}$\\
    $\du = \cos{x}\dx$\\
    $\int\!u^2\,\du$\\
    $= \frac{u^3}{3}$\\
    $\frac{\sin^3{x}}{3}$\\
    $= \frac{\sin^{3}{x}}{3} + C$\\


    \paragraph*{10:\\}
    $\int\!\frac{2x-1}{(x+1)(x^2+9)}\,\dx$\\
    $\frac{1}{10}\int\!\frac{3x+17}{x^2+9}\,\dx - \int\!\frac{3}{10x+10}\,\dx$\\
    $= \frac{1}{10}\int\!\frac{3x}{x^2+9}\,\dx - \int\!\frac{3}{10x+10}\,\dx + \frac{1}{10}\int\!\frac{17}{x^2+9}\dx$\\
    U-sub:\\
    $u = x^2 + 9, \du = 2xdx, x\dx = \frac{\du}{2}$\\
    $= \frac{1}{10}\int\!\frac{3}{2u}\,\dx - \int\!\frac{3}{10x+10}\,\dx + \frac{1}{10}\int\!\frac{17}{x^2+9}\dx$\\
    $= \frac{3}{20}\ln{\abs{x^2 + 9}} - \int\!\frac{3}{10x+10}\,\dx + \frac{1}{10}\int\!\frac{17}{x^2+9}\dx$\\
    $= \frac{3}{20}\ln{\abs{x^2 + 9}} - \int\!\frac{3}{10x+10}\,\dx + \frac{17}{10}\int\!\frac{1}{x^2+9}\dx$\\
    U-sub:\\
    $x = 3u$\\
    $\dx = 3\du$\\
    $u = \frac{x}{3}$\\
    $= \frac{3}{20}\ln{\abs{x^2 + 9}} - \int\!\frac{3}{10x+10}\,\dx + \frac{17}{30}\int\!\frac{1}{u^2+1}\dx$\\
    $= \frac{3}{20}\ln{\abs{x^2 + 9}} - \int\!\frac{3}{10x+10}\,\dx + \frac{17}{30}\arctan{u}$\\
    $= \frac{3}{20}\ln{\abs{x^2 + 9}} - \int\!\frac{3}{10x+10}\,\dx + \frac{17}{30}\arctan{\frac{x}{3}}$\\
    $= \frac{3}{20}\ln{\abs{x^2 + 9}} - \frac{3}{10}\int\!\frac{1}{x+1}\,\dx + \frac{17}{30}\arctan{\frac{x}{3}}$\\
    U-sub:\\
    $u = x + 1$\\
    $\du = \dx$\\
    $= \frac{3}{20}\ln{\abs{x^2 + 9}} - \frac{3}{10}\ln{\abs{u}} + \frac{17}{30}\arctan{\frac{x}{3}}$\\
    $= \frac{3}{20}\ln{\abs{x^2 + 9}} - \frac{3}{10}\ln{\abs{x + 1}} + \frac{17}{30}\arctan{\frac{x}{3}} + C$\\








\thispagestyle{fancy}

\end{document}

